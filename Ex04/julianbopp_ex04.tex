% Continuous Optimization
% Aurelien Lucchi
\documentclass{article}

\usepackage{amsmath,amssymb,amsthm}
\usepackage{algorithm}
\usepackage{algorithmic}
\usepackage{mathtools}
\usepackage[usenames,dvipsnames,table,xcdraw]{xcolor}
\usepackage{enumerate}
\usepackage{graphicx}
\usepackage{hyperref}
\usepackage[margin=0.5in]{geometry}
\usepackage[round]{natbib}


\definecolor{coolblack}{rgb}{0.0, 0.2, 0.3}

\newcommand{\HomeworkV}[3]{%
  \textcolor{coolblack}{\textbf{Homework #1 (#2):}}
  \vspace{5mm}
  \par
  #3
  \par
  \vspace{5mm}
}


\newcommand{\Homework}[3]{
   \pagestyle{myheadings}
   \thispagestyle{plain}
   \newpage
   \setcounter{page}{1}
   \noindent
   \begin{center}
   \framebox{
      \vbox{\vspace{2mm}
    \hbox to 6.28in { {\bf Continuous Optimization
	\hfill Srping 2023} }
       \vspace{4mm}
       \hbox to 6.28in { {\Large \hfill Homework #1: #2  \hfill} }
       \vspace{2mm}
       %\hbox to 6.28in { {\it Lecturer: #3 \hfill Scribes: #4} }
       \hbox to 6.28in { {\it Lecturer: #3 \hfill} }
      \vspace{2mm}}
   }
   \end{center}
   \markboth{Homework #1: #2}{Homework #1: #2}

   \vspace*{4mm}
}


\newcommand{\norm}[1]{\left\lVert#1\right\rVert}
\newcommand{\minsub}[1]{\operatorname*{min}_{#1}}
\newcommand{\argminsub}[1]{\operatorname*{argmin}_{#1}}
\newcommand{\twovector}[2]{\begin{pmatrix} #1 \\ #2 \end{pmatrix}}
\newcommand{\vect}[1]{\mathbf{#1}}
\newcommand{\define}{\coloneqq}
\newcommand{\st}{\textnormal{s.t.}\ }
\newcommand{\set}[1]{\left\{ #1 \right\}}

\begin{document}
\Homework{4}{Constrained Optimization}{Aurelien Lucchi, Student: Julian Bopp}


\HomeworkV{1}{Constrained problem}{
Consider the following 2-dimensional problem
\begin{align*}
\min f(x&,y)\define x(1-y^2) \\
&\st x^2+y^2 = 1.
\end{align*}

\begin{enumerate}
\item Write the stationary and primal feasibility conditions.

\underline{Stationary conditions}:
Define the lagrangian
\begin{align*}
L(x,y,v) &= f(x,y) + v l(x,y)\\
         &= x(1-y^2) + v (x^2+y^2 - 1),
\end{align*}
where $l(x,y) \define x^2+y^2 -1$ encodes the equality constraint.
Find the derivative's w.r.t $x$ and $y$
\begin{align*}
\partial_x L(x,y,v) &= 1-y^2 + 2vx \\
\partial_y L(x,y,v) &= -2yx + 2vy.
\end{align*}
Now the stationary conditions are
\begin{align*}
1-y^2 + 2vx &= 0\\
-2yx + 2vy &= 0.
\end{align*}

\underline{Primal feasibility condition}:
The primal feasibily condition is that the constraint is satisfied, i.e. that
\begin{equation*}
l(x,y) \define x^2 + y^2 - 1 = 0
\end{equation*}


\item Derive the optimal solution $(x^*,y^*)$.

Solving the system of two equations from the stationary conditons we get
\begin{equation*}
\set{(-\frac{1}{2v},0),(v,1-2v^2) }
\end{equation*}
as a set of possible solutions. Combining this with the primal feasibility condition yields
\begin{equation*}
\set{(-1,0),(1,0),(0,-1),(0,1)}
\end{equation*}
as a set of possible solutions. We see that
\begin{align*}
f(-1,0) &=-1 \\
f(1,0) &= 1\\
f(0,-1) &= 0 \\
f(0,1) &= 0,
\end{align*}
and conclude that $(x^*,y^*)\define(-1,0)$ solves the constrained problem.

\end{enumerate}

}

\HomeworkV{2}{KKT problem with two constraints}{
Consider the following 3-dimensional problem
\begin{align*}
&\min f(x,y,z)\define x+y+z \\
&\st x^2-y^2 = 1\ \textnormal{and}\ 2x+z-1 =0.
\end{align*}

\begin{enumerate}
\item Write the stationary and primal feasibility conditions.

\underline{Stationary conditions}:
Define the lagrangian
\begin{align*}
L(x,y,z,v_1,v_2) &= f(x,y,z) + v_1 l_1(x,y,z) + v_2 l_2(x,y,z)\\
         &= x+y+z + v_1(x^2-y^2-1) + v_2(2x+z+1),
\end{align*}
where $l_1(x,y,z) \define x^2-y^2 -1$ and $l_2(x,y,z)\define 2x+z+1$ encode the equality constraints.
Find the derivative's w.r.t $x$ and $y$ and $z$
\begin{align*}
\partial_x L(x,y,z,v_1,v_2) &= 1+2v_1 x + 2v_2 \\
\partial_y L(x,y,z,v_1,v_2) &= 1-2v_1y \\
\partial_z L(x,y,z,v_1,v_2) &= 1+v_2.
\end{align*}
Now the stationary conditions are
\begin{align*}
1+2v_1 x + 2v_2 &=0\\
1-2v_1y &=0\\
1+v_2 &=0.
\end{align*}


\underline{Primal feasibility conditions}:
The primal feasibily conditions are that the equality constraints are satisfied, i.e. that
\begin{alignat*}{3}
l_1(x,y,z) &\define x^2 - y^2 - &1 = 0\\
l_2(x,y,z) &\define 2x+z-&1 = 0
\end{alignat*}

\item Derive all the optimal solutions.

From the the 3rd equality of the stationary conditions we get $v_2=-1$ and with that we get
$x=y=\frac{1}{2v_1}$. Combining this with the first primal feasibility condition we geometry
\begin{align*}
\left(\frac{1}{2v_1}\right)^2- \left(\frac{1}{2v_1}\right)^2 -1 = 0 \implies -1 = 0,
\end{align*}
which is a contradiction. Therefore no optimal solution exists.

\item Can you comment on the results?

There exists no solution because $f$ is not bounded from below on the constraints. To see this let $x=k\geq1$. To satisfy the first equality constraint we pick $y=-\sqrt{k^2-1}$ and to satisfy the second equality constraint we let $z = 1 - 2k$. We then get $f(x,y,z) = -k-\sqrt{k^2-1} + 1$, which goes to $-\infty$, as $k$ goes to $\infty$.

\end{enumerate}

}



\HomeworkV{3}{Projection onto hyperplane}{
Consider the projection of a vector onto a hyperplane identified by the equation $\vect{Ax} = \vect{b}$, i.e.
\begin{equation}\label{eq:1}
\vect{x} = \textnormal{Proj}_{\vect{Ax=b}}(\vect{y}) = \argminsub{\vect{x:Ax=b}}\frac{1}{2}\norm{\vect{y-x}}^2.
\end{equation}

\begin{enumerate}
\item Write down the Lagrangian corresponding to the constrained problem defined in Eq. \eqref{eq:1}.

We want to minimize the function $f(\vect{x})=\frac{1}{2}\norm{\vect{y-x}}^2$ given the equality constraint $\vect{Ax=b}$. Let $\vect{A} \in \mathbb{R}^{m\times n}$, therefore $\vect{x}\in\mathbb{R}^n$ and $\vect{b}\in\mathbb{R}^m$.
We can encode the equality constraint $\vect{Ax=b}$ as a vector valued function
\begin{equation*}
l: \mathbb{R}^n \to \mathbb{R}^m,\ x \mapsto Ax-b.
\end{equation*}
Now let $\vect{v} = (v_1,\dots,v_m)^\top$ and define the lagrangian
\begin{align*}
L(\vect{x},\vect{v}) &= f(\vect{x}) + \vect{v}^\top l(\vect{x})\\
                     &= \frac{1}{2}\norm{y-x}^2 + \vect{v}^\top (\vect{Ax-b}).
\end{align*}

\item Calculate the optimal value of $\vect{x}$ (using the KKT conditions). Show that
\begin{equation*}
\vect{x = Py + A^\top (AA^\top})^{-1}\vect{b},
\end{equation*}
where $\vect{P}\define \vect{I-A^\top}(\vect{AA^\top}^{-1}\vect{A})$ is a projection matrix.

\underline{Stationary conditions}:
Find the derivative of the lagrangian w.r.t $\vect{x}$
\begin{equation*}
\partial_{\vect{x}}L(\vect{x},\vect{v}) = \vect{x-y} + A^\top v.
\end{equation*}
Therefore the stationary condition is $\vect{x-y+A^\top v} = \vect{0}$, i.e. $\vect{x} = \vect{y-A^\top v}$.

\underline{Primal feasibility conditions}
The primal feasibility conditions are our equality constraints $\vect{Ax=b}$. Using the result obtained from the stationary conditions we get
\begin{align*}
&\vect{A(y-A^\top v)=\vect{b}}\\
\implies &\vect{Ay-AA^\top v} = \vect{b}\\
\implies &\vect{AA^\top v} = \vect{Ay-b}\\
\implies &\vect{v} = (\vect{AA^\top})^{-1}(\vect{Ay-b})
\end{align*}

Plugging this into our expression for $\vect{x}$ we get
\begin{align*}
\vect{x} &= \vect{y} - \vect{A}^\top (\vect{AA^\top})^{-1}(\vect{Ay-b})\\
         &= \underbrace{(\vect{I}-\vect{A}^\top(\vect{AA}^\top)^{-1}\vect{A})}_{\vect{P}\define}\vect{y} + \vect{A}^\top(\vect{AA}^\top)^{-1}\vect{b},
\end{align*}
which is what was to be shown.


\end{enumerate}


}




\HomeworkV{4}{Normal cones}{

Consider the following two sets:
\begin{equation*}
\Omega_\infty \define \set{\vect{x}\in\mathbb{R}^d : \norm{\vect{x}}_\infty\leq 1},
\end{equation*}

and
\begin{equation*}
\Omega_2\define \set{\vect{x}\in\mathbb{R}^d : \norm{\vect{x}}_2\leq 1},
\end{equation*}

\begin{enumerate}
\item Show that $\Omega_\infty$ and $\Omega_2$ are non-empty, convex and closed.

\textcolor{blue}{These results can be shown independent of the norm used. We use $\Omega$ and $\norm{\cdot}$ for both sets and norms.}

\underline{Non-empty}: $\norm{\vect{0}} = 0$ and therefore $\vect{0}\in\Omega$.

\underline{Convex}: Let $\vect{a},\vect{b}\in\Omega$ and $\lambda\in(0,1)$. $\Omega$ is convex if $\lambda\vect{a} + (1-\lambda)\vect{b}\in \Omega$.
\begin{align*}
\norm{\lambda \vect{a} + (1-\lambda)\vect{b}} &\leq \lambda \norm{\vect{a}} + (1-\lambda)\norm{\vect{b}}\\
                                              &\leq \lambda + 1 - \lambda \\
                                              &= 1.
\end{align*}
Therefore $\lambda\vect{a} + (1-\lambda)\vect{b}\in \Omega$ and $\Omega$ is convex. For the first inequality we used the triangle inequality and for the second inequality we used the fact that $\vect{a},\vect{b}\in\Omega$ and therefore have norm less than $1$.

\underline{Closed}: Let $\Omega^c= \mathbb{R}^d\setminus\Omega$. To show that $\Omega$ is closed we show that
$\Omega^c$ is open. To show that a set is open we must show that for every element in the set there exists an open ball that is completely contained in the set. Let $\vect{x}\in\Omega^c$, this implies $\norm{\vect{x}}>1$. Let $\epsilon = \norm{\vect{x}} -1$ and define the open ball $B(\vect{x},\epsilon)$ with center $\vect{x}$ and radius $\epsilon$. Let $\vect{y}\in B(\vect{x},\epsilon)$. Then by the definition of an open ball we have
$\norm{\vect{x-y}}_2<\epsilon$ (here I use the euclidean norm because I assume that we want to show that
$\Omega_\infty,\Omega_2$ are both open in $(\mathbb{R}^d,\norm{\cdot}_2$). Now we show that $\vect{y}\in \Omega^c$. First, by the triangle inequality
\begin{align*}
\norm{\vect{x}} = \norm{\vect{x-y+y}} \leq \norm{\vect{y}} + \norm{\vect{x-y}}.
\end{align*}
From this we get
\begin{align*}
\norm{\vect{y}} &\geq \norm{\vect{x}} - \norm{\vect{x-y}}\\
                &\overset{(*)}{\geq} \norm{\vect{x}} - \norm{\vect{x-y}}_2\\
                &> 1 + \epsilon - \epsilon \\
                &=1.
\end{align*}
$(*)$ This inequality is trivial if $\norm{\cdot} = \norm{\cdot}_2$, but if $\norm{\cdot}=\norm{\cdot}_\infty$ we use the result that $\norm{\vect{x}}_\infty\leq\norm{\vect{x}}_2\, \forall \vect{x}\in\mathbb{R}^d$.
This shows that for every $\vect{x}\in\Omega^c$ there exists an $\epsilon > 0$ such that the open ball centered at $\vect{x}$ with radius $\epsilon$ is contained in $\Omega^c$. Therefore $\Omega^c$ is an open set and hence $\Omega$ is a closed set.


\item Determine the normal cones of $\Omega_\infty$ and $\Omega_2$ for $d=2$ at the point $\vect{x} = (1,0)^\top$.

From the lecture we know that $N_{\Omega_2} = \set{\twovector{\lambda}{0}: \lambda >0}$, independent of the point.
Therefore
\begin{equation*}
N_{\Omega_2}(\vect{x}) = \set{\twovector{\lambda}{0}: \lambda >0}
\end{equation*}

For $N_{\Omega_\infty}(\vect{x})$ we use the definition
\begin{equation*}
N_{\Omega_\infty}(\vect{x}) = \set{\vect{w} : \langle\vect{w},\vect{y-\twovector{1}{0}}\rangle\leq 0,\ \forall\vect{y}\in\Omega_\infty}
\end{equation*}

This means that a point $\vect{w} = (w_1,w_2)^\top$ in $N_{\Omega_\infty}(\vect{x})$ has to satisfy the following inequality for any
$\vect{y} = (y_1,y_2)$ in $\Omega_\infty$
\begin{equation*}
w_1(y_1-1) + w_2y_2 \leq 0.
\end{equation*}

In particular for $\vect{y_+} = (1,1)$ and $\vect{y_-} = (1,-1)$ we get
\begin{equation*}
w_2\leq 0,\ w_2\geq 0 \implies w_2 = 0
\end{equation*}

Therefore the constraints on $w_1$ are
\begin{equation}\label{eq:w}
w_1(y_1-1)\leq 0.
\end{equation}
Since $\vect{y}\in\Omega_\infty$ we know that $y_1\in[0,1]$ and therefore $(y_1-1)\leq 0$. From this and Eq. \eqref{eq:w} we conclude that
$w_1 \geq 0$ and therefore
\begin{equation*}
N_{\Omega_\infty}(\vect{x}) = \mathbb{R}_0^+\times \set{0}.
\end{equation*}


\end{enumerate}
}

\end{document}
