% Continuous Optimization
% Aurelien Lucchi
\documentclass{article}

\usepackage{amsmath,amssymb,amsthm}
\usepackage{algorithm}
\usepackage{algorithmic}
\usepackage{mathtools}
\usepackage[usenames,dvipsnames,table,xcdraw]{xcolor}
\usepackage{enumerate}
\usepackage{graphicx}
\usepackage{hyperref}
\usepackage[margin=0.5in]{geometry}
\usepackage[round]{natbib}


\definecolor{coolblack}{rgb}{0.0, 0.2, 0.3}

\newcommand{\HomeworkV}[3]{%
  \textcolor{coolblack}{\textbf{Homework #1 (#2):}}
  \vspace{5mm}
  \par
  #3
  \par
  \vspace{5mm}
}


\newcommand{\Homework}[3]{
   \pagestyle{myheadings}
   \thispagestyle{plain}
   \newpage
   \setcounter{page}{1}
   \noindent
   \begin{center}
   \framebox{
      \vbox{\vspace{2mm}
    \hbox to 6.28in { {\bf Continuous Optimization
	\hfill Srping 2023} }
       \vspace{4mm}
       \hbox to 6.28in { {\Large \hfill Homework #1: #2  \hfill} }
       \vspace{2mm}
       %\hbox to 6.28in { {\it Lecturer: #3 \hfill Scribes: #4} }
       \hbox to 6.28in { {\it Lecturer: #3 \hfill} }
      \vspace{2mm}}
   }
   \end{center}
   \markboth{Homework #1: #2}{Homework #1: #2}

   \vspace*{4mm}
}


\newcommand{\norm}[1]{\left\lVert#1\right\rVert}
\newcommand{\minsub}[1]{\operatorname*{min}_{#1}}
\newcommand{\argminsub}[1]{\operatorname*{argmin}_{#1}}
\newcommand{\twovector}[2]{\begin{pmatrix} #1 \\ #2 \end{pmatrix}}
\newcommand{\vect}[1]{\mathbf{#1}}
\newcommand{\define}{\coloneqq}
\newcommand{\st}{\textnormal{s.t.}\ }
\newcommand{\set}[1]{\left\{ #1 \right\}}

\newcommand{\Chi}{\mathcal{X}}
\newcommand{\xk}{\vect{x}_k}
\newcommand{\xkk}{\vect{x}_{k+1}}
\newcommand{\xstar}{\vect{x}^*}

\begin{document}
\Homework{5}{Proximal Gradient Descent}{Aurelien Lucchi, Student: Julian Bopp}

\HomeworkV{1}{Proximal operator for quadratics}{
Consider the quadratic function $f(\vect{x})=\frac{1}{2}\vect{x}^\top \vect{Ax}+\vect{b}^\top \vect{x} + \vect{c}$ where $\vect{A}\succcurlyeq 0$.\\

Prove that
\begin{equation}
\text{prox}_{\eta f}(\vect{x}) = (\vect{I} + \eta \vect{A})^{-1} (\vect{x}-\eta \vect{b}).
\end{equation}\\

\underline{Proof}: We start by writing down the definition of the prox operator
\begin{align*}
\text{prox}_{\eta f} &= \argminsub{\vect{u}} \eta f(\vect{u}) + \frac{1}{2}\norm{\vect{u-x}}^2\\
&= \argminsub{\vect{u}} \underbrace{\frac{1}{2}\vect{u^\top u} + \frac{\eta}{2}\vect{u^\top A u} + \vect{u^\top}(\eta \vect{b}-\vect{x}) + \eta\vect{c} + \frac{1}{2}\vect{x^\top x}}_{g(\vect{u})\define}
\end{align*}

The minimum is attained where $\nabla g(\vect{u}) = 0$, because it is a quadratic convex function.

The gradient is given by
\begin{equation*}
\nabla g(\vect{u}) = \vect{u} + \eta \vect{Au} + \eta \vect{b} - \vect{x}.
\end{equation*}

Solving this for $\nabla g(\vect{u}) = 0$ yields
\begin{equation*}
\vect{u} = (\vect{I} + \eta \vect{A})^{-1} (\vect{x}-\eta\vect{b}),
\end{equation*}

where we used that $\vect{A}\succcurlyeq 0,\ \eta > 0$ and therefore $(\vect{I} + \eta\vect{A})^{-1}$ exists, because the sum of SPD matrices is again SPD.
}

\HomeworkV{2}{Projected Gradient Descent}{

We want to minimize a function $f$ over a convex set $\Chi$. To do so, we use projected gradient descent that, starting from $\vect{x_0}\in \Chi$, performs the following updates:
\begin{align*}
\vect{y}_{k+1} &= \xk - \eta\nabla f(\xk)\\
\xkk &= \Pi_\Chi (\vect{y}_{k+1}).
\end{align*}

\begin{enumerate}
\item Prove that for all $\vect{x}\in\Chi,\vect{z}\in\mathbb{R}^d$:
\begin{equation*}
H_\vect{x}(\vect{z})\define(\vect{x} - \Pi_\Chi(\vect{z}))^\top (\vect{z} - \Pi_\Chi(\vect{z}))\leq 0.
\end{equation*}\\

\underline{Proof}: We notice that $\Pi_\Chi(\vect{z}) = \argminsub{u\in\Chi}\norm{u-z}$. If $\vect{z}\in\Chi$ we have $\Pi_\Chi(\vect{z}) = z$ and therefore $H_\vect{x}(\vect{z}) =0$ and the statement holds. Therefore we now assume $\vect{z} \in \mathbb{R}^d\backslash\Chi$.

Let $\vect{x}\in\Chi$. Since $\Chi$ is convex and $\Pi_\Chi(\vect{z})\in \Chi$, the points on the line segment $\lambda\vect{x} + (1-\lambda)\Pi_\Chi(\vect{z})$ for $\lambda\in[0,1]$ are again in $\Chi$. We now use the fact that $\Pi_\Chi(\vect{z})$ is the closest point to $\vect{z}$ in $\Chi$.
\begin{align*}
\norm{\vect{z} - \Pi_\Chi(\vect{z})}^2&\leq\norm{\vect{z} - (\lambda\vect{x} + (1-\lambda)\Pi_\Chi(\vect{z}))}^2\\
&=\norm{\vect{z} - \Pi_\Chi(\vect{z})}^2 - 2\underbrace{\lambda(\vect{x} -\Pi_\Chi(\vect{z}))^\top(\vect{z}-\Pi_\Chi(\vect{z}))}_{H_\vect{x}(\vect{z})}+\lambda^2\norm{\vect{x}-\Pi_\Chi(\vect{z})}^2.
\end{align*}

Subtracting the term on the left on both sides yields
\begin{equation*}
2\lambda H_\vect{x}(\vect{z}) \leq \lambda^2\norm{\vect{x}-\Pi_\Chi(\vect{z})}^2.
\end{equation*}
Asuming $\lambda \neq 0$, dividing by $2\lambda$, and taking the limit as $\lambda$ goes to zero we get the result.

\item If $\xkk = \xk$ after the projected gradient descent update, then $\xk$ is a minimizer of $f$ over $\Chi$.\\

\underline{Proof}: Using the result from 1. and $\vect{z}=\vect{y}_{k+1}$ we get for every $\vect{x}\in\Chi$
\begin{align*}
(\vect{x} - \Pi_\Chi(\vect{y}_{k+1}))^\top(\vect{y}_{k+1} - \Pi_\Chi(\vect{y}_{k+1}))\leq 0
\end{align*}
Taking a look at the update rules of the projected gradient descent we see that this is the same as
\begin{align*}
(\vect{x} - \xkk)^\top(\xk - \eta\nabla f(\xk) - \xkk)\leq 0.
\end{align*}
Now using the fact that $\xk = \xkk$ this becomes
\begin{align*}
(\vect{x} - \xk)^\top(- \eta\nabla f(\xk) )\leq 0.
\end{align*}
This means that $-\nabla f(\xk) \in N_\Chi(\xk)$. Therefore $x_k$ satisfies the first-order optimality condition and is a minimizer of $f$ over $\Chi$.

\end{enumerate}
}



\HomeworkV{3}{Proximal Gradient Descent: Convergence analysis}{
Assume that $f$ is $\beta$-smooth and $\mu$ strongly-convex, then
\begin{equation}
\norm{\xkk - \xstar}^2\leq (1-\eta\mu)^k \norm{\vect{x}_1-\xstar}^2.
\end{equation}

\underline{Proof}: We assume that $\eta\leq \frac{1}{\beta}$. By Lemma 3 (in lecture notes of proximal gradient lecture) we have for all $\vect{x}, \vect{z}\in\mathbb{R}^d$,

\begin{equation*}
f(\vect{x} - \eta G_\eta(\vect{x})) \leq f(\vect{z}) + \langle G_\eta(\vect{x}), \vect{x} - \vect{z}\rangle
- \frac{\eta}{2}\norm{G_\eta(\vect{x})}^2 - \frac{\mu}{2}\norm{\vect{x} - \vect{z}}^2.
\end{equation*}

With $\vect{z} = \xstar,\ \vect{x}=\xk$, we have
\begin{align*}
f(\xk - \eta G_\eta(\xk))-f(\xstar) &\leq \langle G_\eta(\xk), \xk - \xstar\rangle
- \frac{\eta}{2}\norm{G_\eta(\xk)}^2 - \frac{\mu}{2}\norm{\xk - \xstar}^2\\
&= \frac{1}{2\eta}\left(2(\eta G_\eta(\xk))^\top(\xk-\xstar)-\norm{\eta G_\eta(\xk)}^2\right)-\frac{\mu}{2}\norm{\xk - \xstar}^2\\
&= \frac{1}{2\eta}\left(\norm{\xk - \xstar}^2 - \norm{\xk - \xstar - \eta G_\eta(\xk)}^2\right)-\frac{\mu}{2}\norm{\xk - \xstar}^2\\
&= \frac{1}{2\eta}\left(\norm{\xk - \xstar}^2 - \norm{\xkk - \xstar}^2\right)-\frac{\mu}{2}\norm{\xk - \xstar}^2\\
&= \frac{1}{2\eta}\left((1-\eta\mu)\norm{\xk - \xstar}^2 - \norm{\xkk - \xstar}^2\right)
\end{align*}

Since $\xstar$ is the minimizer of $f$, the left side is bounded from below by $0$ and we get
\begin{equation*}
0\leq\frac{1}{2\eta}\left((1-\eta\mu)\norm{\xk - \xstar}^2 - \norm{\xkk - \xstar}^2\right)
\end{equation*}

Rearranging yields
\begin{align*}
\norm{\xkk - \xstar}^2 \leq (1-\eta\mu)\norm{\xk-\xstar}^2.\\
\end{align*}

Applying this inequality to the norm term on the right hand side $k-1$ times yields the final inequality
\begin{align*}
\norm{\xkk - \xstar}^2 \leq (1-\eta\mu)^k\norm{\vect{x}_1-\xstar}^2.\\
\end{align*}

}



\end{document}
